\documentclass[11pt]{article}

\usepackage{listings}
\usepackage{fancyhdr}
\usepackage[margin=.8in]{geometry}
\usepackage{amsmath}
\usepackage{enumitem}
\usepackage{hyperref}

\linespread{1}
\setlength{\parindent}{0pt}
\setlength{\tabcolsep}{20pt}

% ===========================================================================
% Header / Footer
% ===========================================================================

\pagestyle{fancy}
\lhead{\scriptsize  CSC 212: Data Structures and Abstractions - Spring 2018}\chead{}\rhead{\scriptsize Weekly Problem Set \#11}
\lfoot{}\cfoot{\scriptsize \thepage~of~\pageref{r:lastpage}}\rfoot{}
\renewcommand{\headrulewidth}{0.25pt}
\renewcommand{\footrulewidth}{0.25pt}

% ===========================================================================
% ===========================================================================
\begin{document}
\thispagestyle{empty}

% ===========================================================================
\begin{center}
    {\Large\bf CSC 212: Data Structures and Abstractions}\\
    \medskip
    {\Large\bf Spring 2018}\\
    \medskip
    {\Large\bf University of Rhode Island}\\
    \bigskip
    {\Large\bf Weekly Problem Set \#11}
\end{center}

Due Thursday 4/19 before class. Please turn in neat, and organized, answers hand-written on standard-sized paper \textbf{without any fringe}. At the top of each sheet you hand in, please write your name, and ID.
The only library you're allowed to use in your answers is \verb|iostream|.

\section{Binary Search Trees}
\begin{enumerate}
    \item Draw a binary search tree after the following operations steps:
    \begin{enumerate}
        \item Insert: [10, 5, 12, 8, 19, 6, 2, 11, 15, 9, 7]
        \item Remove: [7, 12, 8, 10]
    \end{enumerate}
    
    \item Write a function to delete binary trees. Be sure to remove nodes in the proper order, so that none get orphaned.

    \item Write a recursive function that, given a binary tree, returns a tree of the same shape, where every node's value has been set to it's own depth in the tree. You should use two parameters in your call: a pointer to a node, and current depth. 

    \item Briefly explain the difference between in-order, post-order, and pre-order traversals.
\end{enumerate}

\section{2-3 Trees}
\begin{enumerate}
    \item Draw a 2-3 tree after the following operations steps:
    \begin{enumerate}
        \item Insert: [10, 5, 12, 8, 19, 6, 2, 11, 15, 9, 7]
        \item Remove: [7, 12, 8, 10, 9]
    \end{enumerate}

    \item Write a search algorithm for a 2-3 tree.

    \item Explain the tradeoffs between binary search trees, and 2-3 trees. Justify your answer, citing specific examples where you would use either type of tree.
\end{enumerate}

\section{Optional Problems}
\begin{enumerate}
    \item Implement a binary search tree with all of the following methods: constructor, destructor, insert, search, remove.
\end{enumerate}


\label{r:lastpage}

\end{document}
    
