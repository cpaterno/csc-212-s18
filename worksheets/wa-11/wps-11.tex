\documentclass[11pt]{article}

\usepackage{listings}
\usepackage{fancyhdr}
\usepackage[margin=.8in]{geometry}
\usepackage{amsmath}
\usepackage{enumitem}
\usepackage{hyperref}

\linespread{1}
\setlength{\parindent}{0pt}
\setlength{\tabcolsep}{20pt}

% ===========================================================================
% Header / Footer
% ===========================================================================

\pagestyle{fancy}
\lhead{\scriptsize  CSC 212: Data Structures and Abstractions - Spring 2018}\chead{}\rhead{\scriptsize Weekly Problem Set \#11}
\lfoot{}\cfoot{\scriptsize \thepage~of~\pageref{r:lastpage}}\rfoot{}
\renewcommand{\headrulewidth}{0.25pt}
\renewcommand{\footrulewidth}{0.25pt}

% ===========================================================================
% ===========================================================================
\begin{document}
\thispagestyle{empty}

% ===========================================================================
\begin{center}
    {\Large\bf CSC 212: Data Structures and Abstractions}\\
    \medskip
    {\Large\bf Spring 2018}\\
    \medskip
    {\Large\bf University of Rhode Island}\\
    \bigskip
    {\Large\bf Weekly Problem Set \#11}
\end{center}

Due Thursday 4/19 before class. Please turn in neat, and organized, answers hand-written on standard-sized paper \textbf{without any fringe}. At the top of each sheet you hand in, please write your name, and ID.
The only library you're allowed to use in your answers is \verb|iostream|.

\section{Binary Search Trees}
\begin{enumerate}
    \item Draw a binary search tree after the following operations steps:
    \begin{enumerate}
        \item Insert: [10, 5, 12, 8, 19, 6, 2, 11, 15, 9, 7]
        \item Remove: [7, 12, 8, 10]
    \end{enumerate}
    
    \item Write a function to delete binary trees. Be sure to remove nodes in the proper order, so that none get orphaned.

    \item Assume nodes in a BST contain 4 data members: {\it data, depth, left, right}.  Write a recursive function that, given a pointer to the root of a BST, will update every node's {\it depth} to it's own depth in the tree. 

    \item Briefly explain the difference between in-order, post-order, and pre-order traversals.

    \item Implement a binary search tree with all of the following methods: constructor, destructor, insert, search, remove.

    \item Let $T$ be a full k-ary tree, where $k=2$ (a.k.a. {\it binary tree}), with $n$ nodes.  Let $h$ denote the height of $T$.
    \begin{enumerate}
        \item What is the minimum number of leaves for $T$?  Justify your answer.
        \item What is the maximum number of leaves for $T$?  Justify your answer.
        \item What is the minimum number of internal nodes for $T$?  Justify your answer.
        \item What is the maximum number of internal nodes for $T$?  Justify your answer.
    \end{enumerate}

    \item Give a $O(n)$ time algorithm for computing the \verb|height| of the tree, where $n$ is the number of nodes.

    \item Show that the maximum number of nodes in a binary tree of height $h$ is $2^{h+1}-1$.
\end{enumerate}

\label{r:lastpage}

\end{document}
    
