\documentclass[11pt]{article}

\usepackage{listings}
\usepackage{fancyhdr}
\usepackage[margin=.8in]{geometry}
\usepackage{amsmath}
\usepackage{enumitem}

\linespread{1.2}
\setlength{\parindent}{0pt}
\setlength{\tabcolsep}{15pt}

% ===========================================================================
% Header / Footer
% ===========================================================================
\pagestyle{fancy}
\lhead{\scriptsize  CSC 212: Data Structures and Abstractions - Spring 2018}\chead{}\rhead{\scriptsize Weekly Problem Set \#7}
\lfoot{}\cfoot{\scriptsize \thepage~of~\pageref{r:lastpage}}\rfoot{}
\renewcommand{\headrulewidth}{0.3pt}
\renewcommand{\footrulewidth}{0.3pt}

% ===========================================================================
% ===========================================================================
\begin{document}
\thispagestyle{empty}

% ===========================================================================
\begin{center}
    {\Large\bf CSC 212: Data Structures and Abstractions}\\
    \medskip
    {\Large\bf Spring 2018}\\
    \medskip
    {\Large\bf University of Rhode Island}\\
    \bigskip
    {\Large\bf Weekly Problem Set \#7}
\end{center}

Due Thursday 3/8 before class. Please turn in neat, and organized, answers hand-written on standard-sized paper \textbf{without any fringe}. At the top of each sheet you hand in, please write your name, and ID.
The only library you're allowed to use in your answers is \verb|iostream|.

\begin{enumerate}
    \item Write a recursive function that performs binary search on an array of positive integers. Your function must match the prototype: \verb|int bin_search(const int target, const int* arr, int n);|. If there is no match you should return -1.

    \item List each recursive call to your function given the following starting point: 
    
    \verb|bin_search(2, [1,2,3,4,5,6,7,8], 8);| (you should end up with a list of calls that end when the target value is found)

    \item Again, list each recursive call from the following starting point: 
    \verb|bin_search(10, [5, 7, 8, 9], 4);|

    \item Modify your binary search to return the number closest in value to the target (also known as \'approximate search\'). 

    \item Without looking at your previously written code, write a recursive mergesort function.

    \item Why is mergesort preferrable to insertion sort on large datasets?

    \item Why is insertion sort preferrable on smaller datasets?

    \item Is mergesort stable? Why?

\end{enumerate}

The following is considered optional.

\begin{enumerate}
    \item Describe the mechanism behind binary interpolation search, and modify your search to use the approximation technique described.
\end{enumerate}

\label{r:lastpage}

\end{document}
    
