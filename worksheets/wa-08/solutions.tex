\documentclass[11pt]{article}

    \usepackage{listings}
    \usepackage{fancyhdr}
    \usepackage[margin=.8in]{geometry}
    \usepackage{amsmath}
    \usepackage{enumitem}
    \usepackage{hyperref}
    
    \linespread{1.2}
    \setlength{\parindent}{0pt}
    \setlength{\tabcolsep}{15pt}
    
    % ===========================================================================
    % Header / Footer
    % ===========================================================================
    
    \pagestyle{fancy}
    \lhead{\scriptsize  CSC 212: Data Structures and Abstractions - Spring 2018}\chead{}\rhead{\scriptsize Weekly Problem Set Solutions \#8}
    \lfoot{}\cfoot{\scriptsize \thepage~of~\pageref{r:lastpage}}\rfoot{}
    \renewcommand{\headrulewidth}{0.3pt}
    \renewcommand{\footrulewidth}{0.3pt}
    
    % ===========================================================================
    % ===========================================================================
    \begin{document}
    \thispagestyle{empty}
    
    % ===========================================================================
    \begin{center}
        {\Large\bf CSC 212: Data Structures and Abstractions}\\
        \medskip
        {\Large\bf Spring 2018}\\
        \medskip
        {\Large\bf University of Rhode Island}\\
        \bigskip
        {\Large\bf Weekly Problem Set \#8}
    \end{center}
    
    \section{Recurrences}
    \begin{enumerate}
        \item Find a closed-form equivalent of the following recurrences:
        \begin{enumerate} 
            % http://jeffe.cs.illinois.edu/teaching/algorithms/notes/99-recurrences.pdf (2.1)
            \item The Towers of Hanoi:
            $$T(0) = 0; T(n) = 2T(n-1) + 1$$
            \begin{equation} \label{eq1}
                \begin{split}
                    T(n) & = 2T(n-1) + 1 \\
                         & = 2(2T(n-2) + 1) + 1 \\
                         & = 4T(n-2) + 3 \\
                         & = 4(2T(n-3) + 1) + 3 \\
                         & = 8T(n-3) + 7 \\
                         & = \ldots
                \end{split}
            \end{equation}
            This pattern can be written as follows: $$T(n) = 2^kT(n-k) + (2^k - 1)$$ 
            Unrolling n times would yield: $T(n) = 2^nT(0) + (2^n - 1)$
            Plugging in the base case $T(0) = 0$ gives us $T(n) = 2^n - 1$
        
            % http://jeffe.cs.illinois.edu/teaching/algorithms/notes/99-recurrences.pdf (2.3)
            \item The Merge Sort:
            $$T(1) = 1; T(n) = 2T(\frac{n}{2}) + n$$
            \begin{equation}
                \label{eq2}
                \begin{split}
                    T(n) & = 2T(\frac{n}{2} + n) \\
                         & = 2(2T(\frac{n}{4}) + \frac{n}{2}) + n \\
                         & = 4T(\frac{n}{4}) + 2n \\
                         & = 8T(\frac{n}{8}) + 3n \\
                         & = \ldots
                \end{split}
            \end{equation}
            This pattern can be written as follows: $$T(n) = 2^kT(\frac{n}{2}) + kn$$
            Becoming trivial when $\frac{n}{2^k} = 1$ or $k = \log_{2}n$
            Putting it all together: $$T(n) = nT(1) + n \log_{2}n = n \log_{2}n + n$$
        
            % https://docs.google.com/document/d/1Z2jiWIVzD9tF6SI6Oxez9M6DYd57HuOYNzvFSCSbufk/edit# (Recurrences > 1 > c)
            \item Generic:
            $$T(0) = 1; T(n) = T(n - 1) + 2^n$$
            \begin{equation}
                \label{eq3}
                \begin{split}
                    T(0) & = 1 \\
                    T(1) & = T(1 - 1) + 2^1 = T(0) + 2 = 3 \\
                    T(2) & = T(2 - 1) + 2^2 = T(1) + 4 = 7 \\
                    T(3) & = T(3 - 1) + 2^3 = T(2) + 8 = 15 \\
                    T(4) & = T(4 - 1) + 2^4 = T(3) + 16 = 31 \\
                    T(5) & = \ldots \\
                         & = 1 + 3 + 7 + 15 + 31 + \ldots \\
                         & = (2^1 - 1) + (2^2 - 1) + (2^3 - 1) + (2^4 - 1) + \ldots \\
                         & = \sum_{i=0}^{n} (2^{n+1} - 1) \\
                         & = \Theta(2^{n})
                \end{split}
            \end{equation}
        
            % https://docs.google.com/document/d/1Z2jiWIVzD9tF6SI6Oxez9M6DYd57HuOYNzvFSCSbufk/edit# (Recurrences > 1 > e)
            \item Generic:
            $$T(1) = 1; T(n) = T(\frac{n}{3}) + 1$$
            \begin{equation}
                \label{eq4}
                \begin{split}
                    T(n) & = T(\frac{n}{3}) + 1 \\
                    T(\frac{n}{3}) & = T((\frac{n}{3}) / 3) + 1 + 1 = T(\frac{n}{9}) + 2 \\
                    T(\frac{n}{9}) & = T(\frac{n}{27}) + 3 \\
                    T(\frac{n}{27}) & = T(\frac{n}{81}) + 4 \\
                    & = T(\frac{n}{3^k}) + k \\
                    \text{Finding constants: } & \frac{n}{3^k} = 1 \\
                    & n = 3^k \\
                    & k = log_3 n \\
                    & = \sum_{i=1}^k 1 + \log_3 n \\
                    & = \Theta(\log_3 n)
                \end{split}
            \end{equation}
            
        \end{enumerate}
    
    \end{enumerate}
    \section{Merge Sort} 
    \begin{enumerate}
    
        \item Determine the running-time of merge sort for a) sorted input; b) reverse-ordered input; c) random input; d) all identical input. Justify your answers.
        \begin{enumerate}
            \item Merge Sort is guaranteed  $O(n \log n)$ for all cases. The natural variant supports $O(n)$ for already sorted inputs.
        \end{enumerate}
    
    \end{enumerate}
    
    The following is considered optional.
    
    \begin{enumerate}
        
        \item Research and implement Tim Sort. \href{https://en.wikipedia.org/wiki/Timsort}{A link about Tim Sort}
    
        % https://www2.cs.arizona.edu/classes/cs345/summer14/files/reccurence_practice.pdf
        \item Find a closed-form equivalent of the following recurrence:
        $$f(1) = 3; f(n) = f(\frac{n}{2}) + 1$$
    
    \end{enumerate}
    
    \label{r:lastpage}
    
    \end{document}
        
    