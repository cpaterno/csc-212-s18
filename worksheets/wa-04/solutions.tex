\documentclass[11pt]{article}

    \usepackage{listings}
    \usepackage{fancyhdr}
    \usepackage[margin=.8in]{geometry}
    \usepackage{amsmath}
    \usepackage{enumitem}
    
    \linespread{1.3}
    \setlength{\parindent}{0pt}
    
    % ===========================================================================
    % Header / Footer
    % ===========================================================================
    \pagestyle{fancy}
    \lhead{\scriptsize  CSC 212: Data Structures and Abstractions - Spring 2018}\chead{}\rhead{\scriptsize Weekly Problem Set \#4}
    \lfoot{}\cfoot{\scriptsize \thepage~of~\pageref{r:lastpage}}\rfoot{}
    \renewcommand{\headrulewidth}{0.3pt}
    \renewcommand{\footrulewidth}{0.3pt}
    
    % ===========================================================================
    % ===========================================================================
    \begin{document}
    \thispagestyle{empty}
    
    % ===========================================================================
    \begin{center}
        {\Large\bf CSC 212: Data Structures and Abstractions}\\
        \medskip
        {\Large\bf Spring 2018}\\
        \medskip
        {\Large\bf University of Rhode Island}\\
        \bigskip
        {\Large\bf Weekly Problem Set \#4}
    \end{center}
    
    Due Wednesday 2/21 before lab. Please turn in neat, and organized, answers hand-written on standard-sized paper \textbf{without any fringe}. At the top of each sheet you hand in, please write your name, and ID.
    The only library you're allowed to use in your answers is \verb|iostream|.
    
    \begin{enumerate}[leftmargin=*]
    
        % R-1.7
        \item Rank the following functions by their asymptotic growth rate in ascending order.  In your solution, group those functions that are big-Theta of one another (all $\log$ functions are base 2):
        \begin{equation*}
            \begin{array}{ccccc}
                1/n & & & & Sub-Constant \\
                2^{100} & & & & Constant \\
                \log \log n & \sqrt{\log n} & \log^2 n & & Logarithmic \\
                n^{0.01} & \lceil\sqrt{n}\rceil & & & Square Root \\
                2^{\log n} & & & & Linear \\
                6 \cdot n\log n & & & & Linearithmic \\
                4n^{3/2} & 4^{\log n} & n^2\log n & n^3 & Polynomial (c>1) \\
                4^n & 2^{2^n} & & & Exponential
            \end{array}           
        \end{equation*}
        https://www.desmos.com/calculator/svspzsyq1x
    
        % R-1.3
        \item Algorithm \verb|algo1| uses $10n\log n$ operations, while algorithm \verb|algo2| uses $n^2$ operations.  What is the value of $n_0$, such that \verb|algo1| is better than \verb|algo2| for $n\ge n_0$.
        \begin{enumerate}
            \item At 58.77 the two graphs converge, and $n^2$ becomes the slower algorithm.
        \end{enumerate}
    
        % R-1.12 R-1.14 R-1.15
        \item For each of the following, give both a big-Oh characterization in terms of $n$, and an exact characterization (count additions and multiplications):
            \begin{enumerate}
            \item
            \begin{verbatim}
            EX: For the following, the big-Oh characterization is: O(n), 
            the exact characterization is n.
            s = 1
            for i = 1 to n do
                s = s * i
            \end{verbatim}
            \item Exactly $4n$, $O(n)$
            \begin{verbatim}
            s = 1
            for i = 1 to 4n do
                s = s * i
            \end{verbatim}
            \item Exactly $n^3$, $O(n^3)$
            \begin{verbatim}
            s = 1
            for i = 1 to n*n*n do
                s = s * i
            \end{verbatim}
            \item Exactly $4n * \frac{n-1}{2}$, $O(n^2)$
            \begin{verbatim}
            s = 0
            for i = 1 to 4n do
                for j = 1 to i do
                    s = s + i
            \end{verbatim}
            \item Exactly $n^2 * \frac{n-1}{2}$, $O(n^3)$
            \begin{verbatim}
            s = 0
            for i = 1 to n*n do
                for j = 1 to i do
                    s = s + i
            \end{verbatim}
            \item Exactly $n*n*n$, $O(n^3)$
            \begin{verbatim}
            s = 1
            for i = 1 to n do
                for j = 1 to n do
                    for k = 1 to n do
                        s = s * i
            \end{verbatim}
        \end{enumerate}
    
        % C-1.8
        \item Suppose you run two algorithms, \verb|P| and \verb|Q|, on many randomly generated data sets.  \verb|P| is an $O(n \log n)$-time algorithm and \verb|Q| is an $O(n^2)$-time algorithm.  After your experiments you find that if $n<100$, \verb|Q| actually runs faster, and only when $n\ge 100$, \verb|P| is faster.  Explain why this scenario is possible, including numerical examples.
        \begin{enumerate}
            \item This occurs when the constants attached to the $n \log n$ algorithm are too high. Since we are given big-o notation, we do not get the exact runtime analysis. For example, if the logarithmic equation has a 15.1 constant scalar, then it is slower until n > 100. 
        \end{enumerate}
    
    \end{enumerate}
    
    The following is considered optional:
    \begin{enumerate}
        
        \item Given an array \verb|A|, of $n$ integers, describe a method to find the longest subarray of \verb|A| such that all the numbers in that subarray are in sorted order.  What is the running time of your algorithm?
        \begin{enumerate}
            \item To solve this problem start with 4 helpers: global start index, global end index, current start index, current end index. For each element in the array, if the element before it was smaller than itself, increment current. Once this condition is not true, check current against global; if current is larger replace global, otherwise reset current start and end to the current position.
        \end{enumerate}
    
    \end{enumerate}
    
    \label{r:lastpage}
    
    \end{document}
        
    